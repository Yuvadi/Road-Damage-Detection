\begin{abstract}
The deterioration of road surfaces, manifesting as cracks (transverse, longitudinal, alligator, etc.) and potholes, imposes a heavy burden on municipal budgets every year. Traditional manual inspection is labor-intensive and lacks standardization. Emerging technologies—such as vehicle sensor-based monitoring and laser imaging—offer some improvements but are also subject to limitations: sensor data is vehicle-dependent and needs consistent maintenance, while laser imaging is still labor-intensive and not practical for frequent surveys. As a result, computer vision-based crack detection using machine learning has become a very active area of research. However, many existing models are computationally complex, hence demanding high-end hardware. This thesis addresses this challenge by developing a lightweight yet accurate model suitable for deployment on edge devices like dashcams. This allows for real-time crack detection and reporting to local authorities for further analysis and repair. YOLOv11n-OBB has been used as the base model and it has been trained on a custom dataset which combines the RDD2022 dataset with an additional pothole dataset from Kaggle for 450 epochs. The trained model achieves a mean Average Precision (mAP50) of 76.819\% for detecting the following types of cracks: alligator, transverse, and longitudinal cracks, as well as potholes. The original model runs at 9.56 FPS on a Raspberry Pi 5 8GB, while at 10\% pruning gives 10 FPS with 75.81\% mAP50, and at 20 pruning yields 12.5 FPS at 74.5 \% mAP50. In this line, the research showed real-time, edge-based road defect detection is feasible for proactive infrastructure maintenance.
\end{abstract}